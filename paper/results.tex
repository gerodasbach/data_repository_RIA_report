\begin{section}{Results and Discussion}
\autoref{tab_3} shows our results.

\textbf{Specification 1.}
At a high explanatory power ($R^2 = 64\%$), all standard gravity variables except population are statistically significant and resemble the expected signs. We find that the impact of population on bilateral trade is positive for importers and negative for exporters. According to \cite{wla_2021}, this could be due to firstly, larger exporter populations implying larger domestic markets, richer resource endowment and more diversified outputs, as well as less dependence on international specialisation. Secondly, a larger population in an importing country could enhance competition between imported and domestic goods and compensate exporters for the cost of sales activities abroad.

We obtain high average treatment effects of trade relative to the "normal" level. The coefficient $\gamma_1$ shows an increase of trade within the ACFTA of $(e^{1.326}-1)=276\%$. The increase in exports between ACFTA countries and third countries relative to the natural level is ($\delta_1$) 199\%. Imports between ACFTA countries and third countries ($\delta_2$) increase by 31\%. Thus, this model confirms our hypothesis of pure trade creation (compare \autoref{fig_2}). Furthermore, trade flows within all other FTAs except the ACFTA ($otherRTA$) increase by 109\%.

Contrary to our OLS results, \cite{wla_2021}'s pooled OLS coefficients for $\gamma_1$, $\delta_1$ and $\delta_2$ are of the other sign. However, their fixed effects model results show positive signs for all three coefficients. The magnitude of the increase in trade flows (228\%, 272\% and 147\%) they obtain with OLS fixed effects differs only slightly from our OLS results for $\gamma_1$, $\delta_1$ and $\delta_2$. \cite{wla_2021} explain their confirmation of the pattern of pure trade creation (see \autoref{fig_2}) with the expansion of global value chains in the ASEAN region. They claim that intra-industry trade within ACFTA has generated significant momentum for world trade overall. Furthermore, they point out the development and harmonization of product standards as crucial for trade-enhancement in the context of regional trade agreements. 

\textbf{Specification 2.}
The benefit from multilateral resistance terms is that we can put trade barriers within ACFTA into perspective to average trade barriers that ACFTA faces with all of their other FTA trading partners. Comparing $\gamma_1$ and the coefficient of $otherRTA$ to the results of specification 1 allows us to obtain an idea about the importance of multilateral resistance in numbers. Indeed, at high explanatory power ($R^2 = 74\%$) we obtain different results. While $\gamma_1$ becomes statistically insignificant, the $otherRTA$ effect remains significant and positive (+75\%). As in specification 1, our results for the $otherRTA$ coefficient contradict \cite{ksy2018}. Investigating the effects of FTAs on tariffs of non-member countries, they hypothesize that the formation of an FTA leads members to reduce their exports to the rest of the world. We find that the volume of trade within other FTA members increases.

\textbf{Specification 3.}
Using a PPML estimator, we do not obtain statistically significant results on intra-ACFTA trade flows. The coefficient of $otherRTA$ becomes slightly smaller than in the OLS fixed effects estimation of specification 2 (+48\%). Again, we would reject \cite{ksy2018}'s hypothesis of lower trade volumes within FTA blocs of other countries than our chosen ACFTA members.

\textbf{Specification 4.}
\autoref{fig_3} illustrates the concept behind our results for specification 4. Although clearly not a highly significant decrease of -57\% for the average importer country sharing an FTA with a given FTA member country (\cite{dyz_2014}), we find that the magnitude of trade creation within ACFTA and between ACFTA and its third-FTA trade partners is different at the 95\% confidence level. While the ACFTA bloc witnesses an increase in trade flows by 31\%, trade among FTA partner countries other than ACFTA member states increases by 8\%.  

Unfortunately, our $ACFTAexport$ and $ACFTAimport$ dummies get dropped because of multicollinearity. We still want to mention results of the literature.\footnote{\cite{smz2014}'s $ACFTAexport$ and $ACFTAimport$ dummies do not get dropped due to multicollinearity.} Positive and more significant, \cite{smz2014}'s country-pair fixed effects coefficient $\gamma_1$ shows an increase in trade higher than ours (119\% vs. 31\%). 

In line with our results from specification 1, the authors find evidence for unambiguous trade creation across all three dummies: while exports to non-ACFTA countries rise by 58\%, imports from non-ACFTA countries to the bloc increase by 40\%. Using the PML method, their results for the agricultural and manufacturing sector are able to capture zero trade flows and large levels of heteroscedasticity, well. Their results for trade creation are confirmed for trade within ACFTA and exports from ACFTA for manufactured goods and within manufactures for chemical products, as well as for machinery and transport equipment. Only for imports from countries not part of the ACFTA they find effects of trade diversion. Net trade creation is $130\%$ for manufactured goods, $120\%$ for chemical goods and $36\%$ for machinery and transport equipment. This order of magnitudes in their effects resembles the order of magnitude of our obtained estimates of specification 1. Due to the difficulties of interpreting the magnitude of our naive gravity OLS estimates, we cannot assess the claim of \cite{wla_2021} whether it is plausible that it was intra-industry trade within ACFTA that has boosted the pure trade creation effects they obtain.

\end{section}
