\begin{section}{Conclusion}

We cannot reject our first hypothesis of pure trade creation. However, we have to acknowledge that our results are based upon a naive OLS gravity regression. Comparing our results to the literature, we find that our claim is backed up by other studies examining the ACFTA, in particular, for slightly different time frames and data on specific sectors. Although our results resemble the pattern of a sustained increase in trade flows as in \autoref{fig_2}, we lack an econometrically sound approach to evaluate them. A solution could lie in including dummy variables for internal trade flows.

Applying a very similar methodology and specification as \cite{dyz_2014}, we cannot reject our second hypothesis that there is no trade diversion away from third FTA partner countries of the ACFTA. Consulting the literature, and \cite{wla_2021} in particular, we hold it probable that the embeddedness of ACFTA countries in global supply chains is an explanatory factor behind ACFTA not leading to external trade diversion. The limited econometric power of our naive gravity model results makes it impossible to evaluate this claim. Another potential reason behind external trade creation instead of external trade diversion in the case of ACFTA might be the timing of the ACFTA. Ratified just three years after China's WTO accession in 2001, it is likely that our results are biased by this crucial step of China's global trade integration.

Moreover, in feedback with our literature, we take away that trade creation in ACFTA is most pronounced within the bloc, followed by exports from ACFTA. There is less safe evidence for imports from countries outside the ACFTA.

\end{section}