\begin{section}{Conclusion}

We cannot reject our first hypothesis of pure trade creation. However, we have to acknowledge that our results are based upon a naive OLS gravity regression. Comparing our results to the literature, we find that our claim is backed up by other studies examining the ACFTA, in particular, for slightly different time frames and data on specific sectors. Although our results resemble the pattern of a sustained increase in trade flows as in \autoref{fig_2}, we lack an econometrically sound approach to evaluate them. A solution could lie in including internal trade flows to avoid collinearity in $ACFTAexport$ and $ACFTAimport$ dummies with country-time FE, although obtaining consistently measured internal trade is challenging.

We cannot reject our second hypothesis that the ACFTA yields more trade creation than all other FTAs excluding ACFTA countries. Consulting the literature, and \cite{wla_2021} in particular, we hold it probable that the embeddedness of ACFTA countries in global supply chains is an explanatory factor behind ACFTA not leading to external trade diversion. The limited econometric power of our naive gravity model results makes it impossible to evaluate this claim. 

The average efficiency of the ACFTA in terms of trade volume effect is relatively higher than that of the rest of the world's RTAs. Moreover, we find that when taking China out, trade was increasing through FTAs in other parts of the world, as well. Ratified just three years after China's WTO accession in 2001, it is likely that dummy variables on trade diversion (e.g. as in \cite{dyz_2014}) are biased by this crucial step of China's global trade integration. The good news is that our dummy allows us to split within-ACFTA effects from other RTA effects that do not contain China.

Moreover, in feedback with our literature, we take away that trade creation in ACFTA is most pronounced within the bloc, followed by exports from ACFTA. There is less safe evidence for imports from countries outside the ACFTA.

\end{section}
