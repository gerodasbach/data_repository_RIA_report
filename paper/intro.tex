\begin{section}{Introduction}
Since the early 1990s, significant progress has been made in regional integration in the most important economic areas in the world. As bilateral and regional trade liberalization is becoming increasingly prominent, it is crucial to ascertain what implications this may have for world trade.

This study will focus on the ASEAN-CHINA Free Trade Agreement Asian (hereafter named "ACFTA"). ASEAN countries and China have been involved in market integration and have gained fame as the “world factory.” Hence, studying the impact of ACFTA will contribute to empirical evidence on the regional integration and international trade flows. 

Any assessment of the trade effects stemming from the formation of free trade agreements is always accompanied by the concepts of trade creation and trade diversion, which were first introduced by \cite{viner1950}.

Trade creation occurs when new trade arises between member countries due to reduced internal trade barriers. Trade diversion emerges when imports from a higher-cost member country replace imports from a low-cost extra-bloc country because the intra-bloc country has preferential access to the market and does not have to pay tariffs. Thus, the results of trade creation in an improvement in resource allocation presumably have positive welfare effects. Conversely, trade diversion refers to a welfare loss caused by a shift in the origin of a product from an extra-bloc producer (\citeauthor{carrere_2006}, 2006).

The formation of ACFTA helps ASEAN members to access the prosperous Chinese market and fosters economic growth in ASEAN countries. Meanwhile, ACFTA provides China with opportunities to obtain more raw materials to be used in production and helps Chinese enterprises extend their foreign market in Southeast Asia. Generally, ACFTA can be expected to have significant trade creation effects, which means a positive trade effect overall.

This study aims to evaluate the trade creation and diversion effects of the free trade agreements between ASEAN and China. The main research question is by how much trade creation and trade diversion between China and the ASEAN countries change after the introduction of FTA?

The rest of the paper is organised as follows. Section 2 presents a literature review including the evolution of trade between China and ASEAN, theoretical foundations and empirical evidence related to ex-Post assessment of regional trade agreement. Section 3 explains the model specification, our hypotheses and data. Section 4 reports the empirical results. Finally, Section 5 concludes.

\end{section}