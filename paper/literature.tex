\begin{section}{Literature Review}

\begin{subsection}{Evolution of Trade Between China and ASEAN}

China’s accession to the WTO in 2001 marked an important new opportunity to forge closer economic ties around the world. The idea to adopt a ACFTA between ASEAN member states and China was first voiced at the ASEAN-China summit on 6 November 2001. On 4 November 2002, China and ASEAN Countries signed the New Framework Agreement on Comprehensive Economic Cooperation (hereafter named "the Agreement") which formed the legal basis for the creation of the ACFTA (\cite{asean_2002_1}). In order to interpret the evolution of trade flows shown in \autoref{fig_1}, it is worthwhile to consult the Agreement. Policymakers aimed at stimulating trade between ASEAN and China through the reduction of both tariff- and non-tariff barriers.

\begin{figure}[H]
	\centering
	\includegraphics[height=18cm,width=\textwidth]{figure_1.png}
	\caption{\small{Evolution of Exports and Imports of ASEAN Countries with China over Time. Grouped by ASEAN6 and ASEAN4 Countries. \textit{Source:} UN Comtrade Import Data Retrieved in \cite{cepii-data_2022}.}}
	\label{fig_1}
\end{figure}

A potentially substantial part of the rise in trade flows can be associated with the tariff reduction and elimination program as described in Article 3 of the Agreement (\cite{asean_2002_1}). Article 3 foresees a gradual reduction in tariffs along two major categories. \footnote{See \cite{asean_2002_4}, as well as \cite{asean_2002_2} and \cite{asean_2002_3} for details.} \autoref{tab_1} shows that, depending on whether products were listed in the Normal or the Sensitive Track, applied MFN tariffs were to be reduced at different paces, between two different sets of countries. The first set of countries were the earlier ASEAN 6 member states Brunei, Indonesia, Malaysia, Philippines, Singapore and Thailand. The second set of countries were Cambodia, Lao, Myanmar and Vietnam ("ASEAN 4"). As seen, for ASEAN 6 and China, products listed in the Normal Track were to reduce or eliminate their respective applied MFN tariff rates gradually over a period from 1 January 2005 to 2010. In the case of the four newer ASEAN members, the period was from 1 January 2005 to 2015. ASEAN 6 and China were to reduce the applied MFN tariff rates of tariff lines placed in their respective Sensitive Track to 20\% not later than 1 January 2012. These tariff rates were then subsequently reduced to 0-5\% not later than 1 January 2018. Cambodia, Lao, Myanmar and Vietnam were to reduce the applied MFN tariff rates of tariff lines placed in their respective Sensitive Lists to 20\% not later than 1 January 2015. These tariff rates were then subsequently reduced to 0-5\% not later than 1 January 2020. 

\begin{table}[H]
\centering
\begin{tabular}{ccccc} 
\hline
\multicolumn{1}{l}{} & \multicolumn{1}{l}{}   & \textbf{Normal Track }                                                                                        & \multicolumn{2}{c}{\textbf{Sensitive Track}}                                                                                                \\ 
\cline{3-5}
\multicolumn{2}{c}{\textbf{Trade Partners}}   & \begin{tabular}[c]{@{}c@{}}\textbf{Applied MFN Tariff }\\\textbf{to be reduced}\\\textbf{until }\end{tabular} & \begin{tabular}[c]{@{}c@{}}\textbf{Applied MFN Tariff }\\\textbf{to be reduced }\\\textbf{to at least (\%)}\end{tabular} & \textbf{until }  \\ 
\hline
\textbf{ASEAN 6}     & \textbf{China}         & 1 January 2010                                                                                                & 20                                                                                                                       & 1 January 2012   \\
\textbf{ASEAN 6}     & \textbf{China}         & \multicolumn{1}{l}{}                                                                                          & 5                                                                                                                        & 1 January 2018   \\
\textbf{ASEAN 4}     & \textbf{China}         & 1 January 2015                                                                                                & 20                                                                                                                       & 1 January 2015   \\
\textbf{ASEAN 4}     & \textbf{China}         & \multicolumn{1}{l}{}                                                                                          & 5                                                                                                                        & 1 January 2020   \\
\hline
\end{tabular}
\caption{\small{ACFTA Tariff Liberalisation Along Two Different Country Groups and Products. \textit{Source:} \cite{asean_2002_1}.}}
\label{tab_1}
\end{table}

From the annexes of \cite{asean_2002_4} we infer that due to the vast heterogeneity of the tariff policy packages across countries, products and time, attributing trade flow effects to their respective tariff reduction episodes is beyond our scope. While this might sound discouraging at first, there is also another side of the medal. In addition to the tariff reductions, the Agreement foresees a row of further fields of economic integration (\cite{asean_2002_1}). For instance, the parties have agreed on gradually eliminating non-tariff barriers starting 1 January 2005. Furthermore, negotiators took into consideration the different stages of development among ASEAN member states: special conditions in connection to export facilitation were adopted in more recent member states such as Cambodia, Lao and Vietnam, in order for them to catch up in terms of domestic capacity, efficiency and competitiveness.

Given this heterogeneity in trade policies, our aim is to grasp the trade effects of the Agreement as a whole. In our gravity regressions, we account for this by using dummy variables that are meant to capture the entirety of trade liberalisation policies associated with ACFTA.

\end{subsection}

\begin{subsection}{Ex-Post Assessment of Regional Trade Agreements: The Gravity Model}

\begin{subsubsection}{Theoretical Foundation}
A common practice, we believe that starting with a gravity model is best to get an idea about the effects of a regional trade agreement. The structural gravity model describes how bilateral trade flows of country $i$ to country $j$ in period $t$ react to changes in the level of bilateral “freeness” of trade (\cite{mvzj_2019}). Thus, the main advantage of the structural gravity model is that it delivers a tractable framework for trade policy analysis in a multi-country environment. Following \cite{ypl_2016}, we motivate our choice by briefly explaining the structural gravity which our specifications rely upon. 

\begin{equation}\label{eq_1}
    Trade_{ij}=\frac{Y_{i}E_{j}}{Y}\Big(\frac{t_{ij}}{\Pi_{i}P_{j}}\Big)^{1-\sigma}
\end{equation}

The theoretical structural gravity equation \ref{eq_1} governs bilateral trade flows. It can be conveniently decomposed into two terms: (i) a size term $\frac{Y_{i}Y_{j}}{Y}$, (ii) and a trade cost term $\Big(\frac{t_{ij}}{\Pi_{i}P_{j}}\Big)^{1-\sigma}$.
\bigskip
\\
(i) The \textit{size term} consists of the world GDP ($Y$), the GDP of countries $i$ and $j$ ($Y_{i}$ and $Y_{j}$), respectively. Thus, it carries some very useful information regarding the relationship between country size and bilateral trade flows.
\bigskip
\\
(ii) The \textit{trade cost term} captures the total effects of trade costs that drive a wedge between realized and frictionless trade. The trade cost term consists of three components:
\begin{itemize}
\item Bilateral trade cost between partners $i$ and $j$, $t_{ij}$.
\item The structural term $P_{j}$ , coined by \cite{avw2003} as inward multilateral resistance represents importer $j$’s ease of market access.
\item The structural term $\Pi_{i}$, defined as outward multilateral resistance by \cite{avw2003}, measures exporter $i$’s ease of market access.
\end{itemize}

Given the multiplicative nature of the structural gravity equation \ref{eq_1}, and assuming that it holds in each period of time $t$, it is possible to log-linearize it and expand it with an additive error term:
\begin{equation}\label{eq_2}
   \ln{Trade_{ijt}}=\ln{Y_{jt}}+ \ln{Y_{it}}-\ln{Y_{t}}+(1-\sigma)\ln{t_{ijt}}-(1-\sigma)\ln{P_{jt}}-(1-\sigma)\ln{\Pi_{it}}+\varepsilon_{ijt}
\end{equation}
The standard practice suggested in the literature is to proxy for the
bilateral trade cost term, $(1-\sigma)\ln{t_{ijt}}$, by using a series of observable variables most of which have become standard covariates in empirical gravity specifications \cite{ypl_2016}:
\begin{equation}\label{eq_3}
(1-\sigma)\ln{t_{ijt}}=\beta_{1}\ln{Dist_{ij}} + \beta_2 Lang_{ij} + \beta_3 Contig_{ij} + \beta_4RTA_{ijt}+\varepsilon_{ijt}
\end{equation}

Inserting equation \ref{eq_3} into equation \ref{eq_2} yields the core idea behind our specifications. In order to obtain reliable estimates, we address several estimation techniques (see Section 3). 


\end{subsubsection}

\begin{subsubsection}{Empirical Review: Trade Creation, Trade Diversion or Both?}

\begin{figure}[H]
	\centering
	\includegraphics[width=\textwidth]{figure_2.png}
	\caption{\small{Trade Creation and Trade Diversion? Trade Flows Relating to ACFTA over Time. \textit{Source:} BACI Trade Flows Retrieved in \cite{cepii-data_2022}. Missing BACI Trade Flow Data were Replaced by UN Comtrade Import Data.}}
	\label{fig_2}
\end{figure}

This part intends to provide a brief survey of those papers closest to our methodology and regional trade integration episode. Conceptually, our aim to obtain trade integration effects goes back to \cite{viner1950}. Extending Viner's core idea of different trade effects, depending on whether a country is part of an integration episode or not, we model three different sets of FTA dummy variables representing trade creation and diversion effects in terms of export and import.\footnote{It was \cite{endoh1999} who first formalized this idea.} As our study intends to add to the grand debate around Vinerian trade creation and trade diversion in the literature on the ACFTA, we focus on studies that explore the trade impacts of ACFTA on ASEAN member countries, China and their respective third trading partners. 

\cite{smz2014}'s structural gravity framework intends to capture trade creation and trade diversion effects of ACFTA. They test their model on a sample of 31 countries over the period from 1995 to 2010 using UNCTAD export data for agricultural and manufactured goods and within manufactures for chemical products, as well as for machinery and transport equipment. Distinguishing between agricultural and manufacturing goods, they acknowledge the fact that the agreements involve different tariff-reduction schedules over time\footnote{Different provisions apply for agricultural goods (Early Harvest Products) and for manufacturing goods (Normal Track of the Agreement, see Section 2.1)}, and such can have distinctive effects on trade flows. Zooming in allows to examine sectors of particular importance for the overall trade effects.

A second study that uses a structural gravity framework is \cite{wla_2021}. The authors investigate the effects of exports from ACFTA to the 79 countries that together make up 95\% of ASEAN's export volume. In order to differentiate whether the formation of ACFTA has accelerated trade among both ASEAN member countries and non-ASEAN members, they employ three different dummy variables.  This allows them to separate between three different effects of trade creation and diversion, measured by export and import flows both within \textit{ACFTA} countries and between \textit{ACFTA} and \textit{Non-ACFTA} countries (see also \cite{carrere_2006}). First, the authors regress log trade flows on the dummies, controlling for GDP, population, language, distance and border contiguity, landlocked and islands in a pooled OLS regression. Second, they estimate the same specification with random effects. Finally, employing a battery of time-invariant fixed effects, the authors regress log trade flows on the dummies and log GDP and log population. 

Data on trade flows from the CEPII TradeProd and UN COMTRADE database of 2-digit ISIC manufacturing trade and FTAs for 64 countries during the period 1990–2002 at hand, \cite{dyz_2014} employ a structural gravity model to estimate effects of FTAs in terms of trade creation and trade diversion with a PPML estimator. 


\end{subsubsection}

\end{subsection}

\end{section}