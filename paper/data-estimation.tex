\begin{section}{Methodology and Data }

\begin{subsection}{Methodology }

To answer our research question, our specifications need to allow for the possibility of trade creation and trade diversion.

\begin{figure}[H]
	\centering
	\includegraphics[width=\textwidth]{figure_4.png}
	\caption{\small{A Categorisation of Dummy Variables to Measure the Impact of Regional Trade Agreements.}}
	\label{fig_4}
\end{figure}

Our methodology departs from the $RTA$ dummy of \autoref{eq_4}. Our aim is to compare the effects of ACFTA on bilateral trade flows within ACFTA to ACFTA-sourced imports from third countries and ACFTA-directed exports from third countries. Therefore, we make use of three corresponding dummy variables as in \autoref{tab_2}. It is this triad of dummy variables that enables us to obtain evidence on the so-called Viner ambiguity in the context of the ACFTA (see \autoref{tab_2}). Our study is to be viewed as adding robustness to \cite{smz2014} and \cite{wla_2021}. Furthermore, we include a dummy that captures trade flows occurring between all other third countries that are part of an RTA that is different from ACFTA. This helps us to compare the strength of the ACFTA RTA to the strength of all other RTAs in the world. In our specifications, their combined effect is captured by the \textbf{$otherRTA$} dummy.

\textbf{\textbf{$ACFTA_{intra}$}} estimates the impact of the commencement of the agreement on ASEAN-China trade flows. Throughout all specifications, $ACFTA$ takes the value one when countries $i$ and $j$ belong to ACFTA in year $t$ and zero if otherwise. A positive (negative) coefficient $\gamma_1$ represents trade creation (diversion), depending on whether intra-bloc trade is higher (lower) than the normal trade levels due to the trade agreement. 

\textbf{\textbf{$ACFTA_{export}$} }takes a value of one if exporter country $i$ belongs to the ACFTA in year $t$ and destination country $j$ does not belong to the ACFTA countries and zero otherwise. A statistically significant and positive $\delta_1$ coefficient can be interpreted as an export creation effect. It implies that regional trade integration leads to export diversion from ACFTA member countries to non-ACFTA countries. However, a negative $\delta_1$ coefficient signifies an export reduction between member countries and non-member countries, known as the export diversion effect (\cite{carrere_2006}). 

\textbf{\textbf{$ACFTA_{import}$}} takes a value of one if exporter $i$ is a non-member country of ACFTA in year $t$ and destination country $j$ belongs to the ACFTA member countries and zero otherwise. Essentially, a statistically significant and positive $\delta_2$ coefficient captures an import creation effect, showing the expansion of imports from the non-member countries to member countries. Contrarily, a negative $\delta_2$ coefficient indicates the import diversion effect, a decrease in imports from non-member- to member countries (\cite{carrere_2006}).

Our ex-post assessment of the trade effects of ACFTA starts with a naïve regression specification, and considers only international trade flows (i.e. for $i \neq j$). Stepwise, we include additional modifications. Each modification introduces a new feature to the initial specification, aiming at correctly identified estimates of the three ACFTA effects and the RTA-comparison.

\subsubsection*{Specification 1: OLS estimation ignoring multilateral resistance terms}

We first do an OLS estimation of the empirical specification that includes standard gravity variables with panel data with 4-year intervals\footnote{Interval panel data should be employed in order to allow for adjustment in bilateral trade flows in response to trade policy or other changes in trade costs (\cite{ypl_2016})}:

\begin{multline}\label{eq_5}
\ln{Trade_{ijt}} = \alpha_0 + \beta_1 \ln{Y_{it}} + \beta_2 \ln{Y_{jt}} + \beta_3 \ln{Pop_{it}} + \beta_4 \ln{Pop_{jt}}  + \beta_5 \ln{Dist_{ij}} + \beta_6 Lang_{ij} + \beta_7 Contig_{ij} + \\
\gamma_1 ACFTAintra_{ijt} + \gamma_2 otherRTA_{ijt} + \\ 
\delta_1 ACFTAexport_{ijt} + \delta_2 ACFTAimport_{ijt} + \epsilon_{ijt}
\end{multline}

In this basic gravity model, we make aggregate total bilateral trade flows $lnTrade$ from to country $j$ to country $i$ dependent on Gross Domestic Product ($lnY$), population ($lnPop$) and distance ($lnDist$). Moreover, we include the binary dummies common border ($Contig$) and language ($Lang$). 


\subsubsection*{Specification 2: OLS estimation controlling for multilateral resistance terms with fixed effects}

As the theoretical foundation, the multilateral resistance terms $P_{j,t}$ and $Pi_{i,t}$ should be properly controlled in the gravity model; however, the difficulty is that they are not directly observable by the researcher and/or by the policymaker. \cite{Feenstra2016} proposes and demonstrates that multilateral resistance terms should be accounted for by exporter-time and importer-time fixed effects in a dynamic gravity estimation framework with panel data to overcome this challenge. In this way, the exporter-time and importer-time fixed effects will also absorb the size variables ($E_{j,t}$ and $Y_{i,t}$ ) from the structural gravity model as well as all other observable and unobservable country-specific characteristics.

Thus, specification \ref{eq_5} is modified to account for the multilateral resistances as follows.

\begin{multline}\label{eq_6}
\ln{Trade_{ijt}} = \mu_{it} + \phi_{jt} + \beta_5 \ln{Dist_{ij}} + \beta_6 Lang_{ij} + \beta_7 Contig_{ij} + \\ \gamma_1 ACFTAintra_{ijt} + \gamma_2 otherRTA_{ijt} +  \epsilon_{ijt}
\end{multline}

$\mu_{it}$ denotes the vector of exporter-time fixed effects, which accounts for the outward multilateral resistances. Similarly, the vector $\phi_{jt}$ denotes the set of importer-time fixed effects to capture the inward multilateral resistances. No constant term is included in the presence of the fixed effects.


\subsubsection*{Specification 3: PPML estimation controlling for multilateral resistance terms with fixed effects}

Specification \ref{eq_6} is re-formulated in multiplicative form and re-estimated by applying the PPML estimator to the same sample (with international trade only).

\begin{multline}\label{eq_7}
Trade_{ijt} = exp \Big(\mu_{it} + \phi_{jt} + \beta_5 lnDist_{ij} + \beta_6 Lang_{ij} + \beta_7 Contig_{ij} \\ 
+ \gamma_1 ACFTAintra_{ijt} + \gamma_2 otherRTA_{ijt} \Big) \epsilon_{ijt}
\end{multline}

\bigskip
Applying the PPML estimator to the gravity model is justified on various grounds. First, the PPML estimator expressed in a multiplicative form, accounts for heteroscedasticity, which often plagues trade data (\cite{silva2006log}). Second, for the same reason, the PPML estimator is able to take advantage of the information contained in the zero trade flows. Third, the additive property of the PPML estimator ensures that the gravity fixed effects are identical to their corresponding structural terms (\cite{fally2015}).

\subsubsection*{Specification 4: Addressing potential endogeneity of FTAs}

Without further ado, trade effects from the formation of ACFTA could have lied fault to a chicken egg problem of reverse causality. Following \cite{ypl_2016}, specification \ref{eq_7} is modified to include country-pair fixed effects $\pi_{ij}$ in addition to the importer-time and exporter-time fixed effects $\mu_{it}$ and $\phi_{jt}$.

\begin{multline}\label{eq_8}
Trade_{ijt} = exp \Big(\mu_{it} + \phi_{jt} + \pi_{ij} + 
\gamma_1 ACFTA_{ijt} + \gamma_2 otherRTA_{ijt}\Big) \epsilon_{ijt}
\end{multline}

Our rationale stems from \cite{bb_2007} who warn of omitted variable bias and simultaneity bias. 
To illustrate a potential source of omitted variable bias, let us suppose two countries with extensive domestic regulations inhibiting trade. Then, once a large expected welfare gain from trade creation via the FTA in sight, the likelihood of the countries selecting into the FTA could have been especially high. 
Simultaneity bias can be illustrated as follows: suppose two countries trade more or less than the gravity-equation-suggested “natural” level. This could be due to an already extensive trading relationship. Contrarily, one can imagine political pressures to avoid trade liberalisation. In both cases, the problem is that the decision to form an FTA is likely influenced by trade relative to the “natural” level. In contrast, recent changes in trade levels are unlikely to influence FTA formations. Hence, looking at recent changes in trade levels between country pairs allows control for the “natural” level of trade. The idea is that unobserved time invariant heterogeneity, hidden in the “natural” level of trade, simultaneously influences the presence of a FTA and the volume of trade. Using panel data we can control for this unobserved time invariant heterogeneity. Exploiting cross-sectional unit variation over time, we can be reassured about identification. 

\end{subsection}



\begin{subsection}{Data}

This study uses unbalanced panel data of bilateral trade flows, GDP, population, distance, geographical, cultural and historical information and a few other group-specific measures from \cite{cepii-data_2022}.

The gravity models include 252 countries covering the period 1995-2019 (24 years). The source of information for aggregated (country-level) bilateral trade flows is BACI trade data downloaded from \cite{cepii-data_2022}. The reason for choosing BACI data for the dependent variable is twofold: first, BACI data are cleaned to exclude re-exports and re-imports. Second, BACI data reconcile reporting differences among countries \cite{gaulier2010baci}. Therefore, BACI data provide a more complete and coherent set of trade flows. We however stick to UN Comtrade import data to replace missing values of BACI trade flows. Again, there are two reasons. First, BACI trade flows stem from the UN Comtrade database. Second, import data is considered more reliable because imports are monitored much more closely than exports by customs administrations, see \cite{ypl_2016}.

Please, find the code and data for reproduction \href{https://github.com/gerodasbach/data_repository_RIA_report.git}{here}.\footnote{Due to the large size of the original data set, it is not part of the github repository. We offer to transfer it and suggest to insert it manually into your \textit{./input} folder. Please contact us per \href{mailto:gero.dasbach@gmail.com}{e-mail}.}

\end{subsection}
\end{section

